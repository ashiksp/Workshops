\documentclass[10pt]{beamer}
\usetheme{Warsaw}
\title{Git - Version Control System}
\author{Ashik Salman}
\institute[UMBC]{Beginner Workshop \\
  ..... \\
  Backend Developer \\
  Chillr, Backwater Technologies \\
  Kochi, Kerala \\

}
\date{September 20, 2016}
\begin{document}

%----Title----%

\begin{frame}[plain]
  \titlepage
\end{frame}
	
%----Slides----%
%-1-%
\begin{frame}
  \begin{center}
    \Huge{What ?}
  \end{center}
\end{frame}

%-2-%
\begin{frame}
  \frametitle{Abstract}
  \textbf{Git Workshop}
  \medskip
  \begin{itemize}
    \item What is Git \& Github.
    \item Git installation and creation of Github account.
    \item Git commands \& Hands-on.
    \item Quick examples.
  \end{itemize}
\end{frame}

%-3-%
\begin{frame}
  \frametitle{Introduction}
  \begin{block}{What is "version control", and why should you care?}
    Version control is a system that records changes to a file or set of files
    over time so that you can recall specific versions later.
  \end{block}
  \medskip
  \pause
  \begin{itemize}
    \item It allows you to revert files back to a previous state.
    \item Revert the entire project back to a previous state.
    \item Compare changes over time, see who last modified something that might be causing a problem.
    \item Who introduced an issue and when, and more.
  \end{itemize}
\end{frame}

%-4-%
\begin{frame}
  \frametitle{Introduction}
  \begin{block}{Distributed Version Control Systems}
    In a Distributed Version Control Systems, clients don’t just check out the latest
    snapshot of the files: \alert<2>{they fully mirror the repository}. Thus if any server dies, and these systems
    were collaborating via it, any of the client repositories can be copied back up to
    the server to restore it. Every clone is really a full backup of all the data.
  \end{block}
  \medskip
  \pause
  \begin{itemize}
    \item Git
    \item Mercurial
    \item Bazaar or Darcs
  \end{itemize}
\end{frame}

%-5-%
\begin{frame}[fragile]
  \frametitle{Getting started - Git}
  \textbf{Installation}
  \begin{exampleblock}{Note !}
    If you understand what Git is and the fundamentals of how it works, then using Git effectively
    will probably be much easier for you.
  \end{exampleblock}
  \textbf{Customizations} \\
  \text{Files: /etc/gitconfig, .git/config, ~/.config/git/config}
  \pause
  \begin{block}{Samples}
    \begin{verbatim}
      $ git config --global user.name "Ashik Salman"
      $ git config --global user.email ashiksp@gmail.com
      $ git config --global core.editor vim
      $ git config --list , Explore more !

      $ git config --global alias.co checkout
      $ git config --global alias.unstage 'reset HEAD --'
    \end{verbatim}
  \end{block}
\end{frame}

%-6-%
\begin{frame}[fragile]
  \frametitle{Initialization}
  \textbf{Getting a Git Repository}
  \begin{block}{Samples}
    \begin{verbatim}
      $ git init
      $ git add *.c
      $ git add LICENSE
      $ git commit -m 'initial project version'

      $ git clone git@github.com:ashiksp/Workshops.git
      $ git clone https://github.com/ashiksp/Workshops.git
    \end{verbatim}
  \end{block}
\end{frame}

%-7-%
\begin{frame}[fragile]
  \frametitle{File states}
  \textbf{Recording Changes to the Repository (States)}
  \begin{itemize}
    \item \alert{Committed}, data is safely stored in your local database.
    \item \alert{Modified}, changed the file but have not committed it to your database yet.
    \item \alert{Staged}, marked a modified file in its current version to go into your next commit snapshot.
  \end{itemize}
  \begin{block}{Samples}
    \begin{verbatim}
      $ git status
      On branch master
      Your branch is up-to-date with 'origin/master'.
      nothing to commit, working directory clean

      $ git status -s  # Short Status
    \end{verbatim}
  \end{block}
\end{frame}

%-8-%
\begin{frame}[fragile]
  \frametitle{Ignoring Files}
  \textbf{Need git to ignore specific file patterns - .gitignore}
  \medskip
  \begin{block}{Samples}
    \begin{verbatim}
      # no .a files
      *.a

      # but do track lib.a, even though you're ignoring
      # .a files above
      !lib.a

      # ignore all files in the build/ directory
      build/

      # ignore doc/notes.txt, but not doc/server/arch.txt
      doc/*.txt

      # ignore all .pdf files in the doc/ directory
      doc/**/*.pdf
    \end{verbatim}
  \end{block}
  \medskip
\end{frame}

%-9-%
\begin{frame}[fragile]
  \frametitle{Basic commands}
  \begin{block}{Viewing Your Staged and Unstaged Changes}
    \begin{verbatim}
      $ git status
      $ git diff
      $ git diff --cached | git diff --staged
    \end{verbatim}
  \end{block}
  \pause
  \begin{block}{Committing Your Changes}
    \begin{verbatim}
      $ git commit
      $ git commit -m "Commit Message"
      $ git commit --amend # Undoing Things
    \end{verbatim}
  \end{block}
\end{frame}

%-10-%
\begin{frame}[fragile]
  \frametitle{Basic commands}
  \begin{block}{Removing Files}
    \begin{verbatim}
      $ git rm filename
      $ git rm --cached README
      $ git rm log/\*.log
    \end{verbatim}
  \end{block}
  \pause
  \begin{block}{Moving Files}
    \begin{verbatim}
      $ git mv README.md README

      $ mv README.md README
      $ git rm README.md
      $ git add README
    \end{verbatim}
  \end{block}
\end{frame}

%-11-%
\begin{frame}[fragile]
  \frametitle{Basic commands}
  \begin{block}{Viewing the Commit History}
    \begin{verbatim}
      $ git log
      $ git log -p
      $ git log --stat
      $ git log --pretty=oneline
      $ git log --pretty=format:"%h - %an, %ar : %s"
      $ git log --graph
    \end{verbatim}
  \end{block}
  \pause
  \begin{block}{Limiting Log Output}
    \begin{verbatim}
      $ git log --since=2.weeks
      $ git log --author 'ashiksp@gmail.com
      $ git log -Sfunction_name'
    \end{verbatim}
  \end{block}
\end{frame}

%-12-%
\begin{frame}[fragile]
  \frametitle{Basic commands}
  \begin{block}{Unstaging a Staged File}
    \begin{verbatim}
      $ git reset HEAD
      $ git reset HEAD CONTRIBUTING.md
    \end{verbatim}
  \end{block}
  \pause
  \begin{block}{Unmodifying a Modified File}
    \begin{verbatim}
      $ git checkout .
      $ git checkout -- CONTRIBUTING.md
    \end{verbatim}
  \end{block}
\end{frame}

%-13-%
\begin{frame}[fragile]
  \frametitle{Basic commands}
  \begin{block}{Working with Remotes}
    \begin{verbatim}
      $ git remote # Showing Your Remotes
      $ git remote -v # Show with urls\
      $ git remote show origin # Inspecting a Remote
      $ Add remote reference:
      $   git remote add remote-name https://url.git
      $ Fetching and Pulling from Your Remotes:
      $   git fetch [remote-name]
      $ Pushing to Your Remotes:
      $   git push origin master
      $ Removing and Renaming Remotes:
      $   git remote rename old-name new-name
      $   git remote rm remote-name
    \end{verbatim}
  \end{block}
\end{frame}

%-x-%
\begin{frame}
  \begin{center}
    \Huge{Thank You}
  \end{center}
\end{frame}

%-x-%
\begin{frame}
	\frametitle{Questions?}
	\begin{center}
	\includegraphics[width=0.35\textwidth]{q2.jpg}
	\end{center}
\end{frame}

%----Slides----%

\end{document}
