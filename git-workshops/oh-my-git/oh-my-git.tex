\documentclass[10pt]{beamer}
\usetheme{Warsaw}
\title{Oh-My-Git}
\author{Ashik Salman}
\institute[UMBC]{Awesome Git tweaks \\
  ..... \\
  Backend Developer \\
  Chillr, Backwater Technologies \\
  Kochi, Kerala \\

}
\date{September 25, 2016}
\begin{document}

%----Title----%

\begin{frame}[plain]
  \titlepage
\end{frame}
	
%----Slides----%
%-1-%
\begin{frame}
  \begin{center}
    \Huge{oh-my-git, what?}
  \end{center}
\end{frame}

%-2-%
\begin{frame}
	\frametitle{Generic View}
	\textbf{Git Tool}
	\begin{center}
	\includegraphics[width=0.42\textwidth]{git-xkcd.png}
	\end{center}
\end{frame}

%-3-%
\begin{frame}
  \begin{center}
    \Huge{Versioning by Patch changes}
  \end{center}
\end{frame}

%-4-%
\begin{frame}[fragile]
  \frametitle{Git with Patches}
  \begin{block}{Add changes by patches}
    \begin{verbatim}
      $ git add -p
      $ git add -i (interactive menu)
    \end{verbatim}
  \end{block}
  \pause
  \begin{block}{Checkout changes by patches(Remove un-staged changes)}
    \begin{verbatim}
      $ git checkout -p
    \end{verbatim}
  \end{block}
  \pause
  \begin{block}{Reset changes by patches(Remove staged changes)}
    \begin{verbatim}
      $ git reset -p
    \end{verbatim}
  \end{block}
\end{frame}

%-5-%
\begin{frame}
  \frametitle{oh-my-git !!!}
  \begin{alertblock}{Problem Scenario}
    I just committed some changes and immediately realized I need to make one small change.
  \end{alertblock}
\end{frame}

%-6-%
\begin{frame}[fragile]
  \frametitle{oh-my-git !!!}
  \begin{example}[Solution]
    \begin{verbatim}
      # make your change
      $ git add .  # or add individual files
      $ git commit --amend
      # follow prompts to change or keep the commit message
      # now your last commit contains that change!
    \end{verbatim}
  \end{example}
  \pause
  \begin{block}{Note !}
    You could also make the change as a new commit and then do rebase -i in order to squash them both together,
    but this is about a million times faster.
  \end{block}
\end{frame}

%-7-%
\begin{frame}
  \frametitle{oh-my-git !!!}
  \begin{alertblock}{Problem Scenario}
   I accidentally committed something to master that should have been on a brand new branch!
  \end{alertblock}
\end{frame}

%-8-%
\begin{frame}[fragile]
  \frametitle{oh-my-git !!!}
  \begin{example}[Solution]
    \begin{verbatim}
      # create a new branch from the current state of master
      $ git branch some-new-branch-name
      # remove the commit from the master branch
      $ git reset HEAD~ --hard
      $ git checkout some-new-branch-name
      # your commit lives in this branch now :)
    \end{verbatim}
  \end{example}
  \pause
  \begin{block}{Note !}
    This doesn't work if you've already pushed to origin, and if you tried other things first,
    you might need to git reset HEAD@{number} instead of HEAD~.
  \end{block}
\end{frame}

%-9-%
\begin{frame}
  \frametitle{oh-my-git !!!}
  \begin{alertblock}{Problem Scenario}
   Ooops..! I accidentally committed to the wrong branch!
  \end{alertblock}
\end{frame}

%-10-%
\begin{frame}[fragile]
  \frametitle{oh-my-git !!!}
  \begin{example}[Solution]
    \begin{verbatim}
     # undo the last commit, but leave the changes available
     $ git reset HEAD~ --soft
     $ git stash
     # move to the correct branch
     $ git checkout name-of-the-correct-branch
     $ git stash pop
     $ git add . # or add individual files
     $ git commit -m "your message here"
     # now your changes are on the correct branch
    \end{verbatim}
  \end{example}
\end{frame}

%-11-%
\begin{frame}[fragile]
  \frametitle{oh-my-git !!!}
  \begin{block}{Note !}
    A lot of people have suggested using cherry-pick for this situation too,
    so take your pick on whatever one makes the most sense to you!
  \end{block}
  \pause
  \begin{example}[Solution]
    \begin{verbatim}
      $ git checkout name-of-the-correct-branch
      # grab the last commit to master
      $ git cherry-pick master
      # delete it from master
      $ git checkout master
      $ git reset HEAD~ --hard
    \end{verbatim}
  \end{example}
\end{frame}

%-12-%
\begin{frame}
  \frametitle{oh-my-git !!!}
  \begin{alertblock}{Problem Scenario}
    I did something terribly wrong, I want to got to a stage where everything worked fine.
  \end{alertblock}
  \pause
  \begin{block}{Why... ?}
    You can use this to get back stuff you accidentally deleted, or just to remove some stuff you
    tried that broke the repo, or to recover after a bad merge, or just to go back to a time when things actually worked.
  \end{block}
\end{frame}

%-13-%
\begin{frame}[fragile]
  \frametitle{oh-my-git !!!}
  \begin{example}[Solution]
    \begin{verbatim}
      $ git reflog
      # you will see a list of every thing you've done in
      # git, across all branches!
      # each one has an index HEAD@{index}
      # find the one before you broke everything
      $ git reset HEAD@{index}
    \end{verbatim}
  \end{example}
  \pause
  \begin{block}{Note !}
    If the work was committed at any point, then it can be recovered from the reflog.
    By default all commits stay alive in the reflog for at least 2 weeks.
  \end{block}
\end{frame}

%-14-%
\begin{frame}[fragile]
  \frametitle{oh-my-git !!!}
  \begin{block}{Total mess-up !}
    \begin{verbatim}
      # Suppose only one branch went terribly wrong...!
      $ git checkout branch-name
      $ git reset HEAD~50 --hard
      $ git pull upstream branch-name
      $ git push origin -f branch-name

      # Delete the repo and clone a fresh one.
      $ cd ..
      $ sudo rm -r messed-git-repo-dir
      $ git clone
      $  https://some.github.url/messed-git-repo-dir.git
      $ cd messed-git-repo-dir
    \end{verbatim}
  \end{block}
\end{frame}

%-15-%
\begin{frame}
  \begin{center}
    \Huge{Git Merge (Conflicts!!!)}
  \end{center}
\end{frame}

%-16-%
\begin{frame}[fragile]
  \frametitle{oh-my-git !!!}
  \begin{block}{Basic Merge !}
    \begin{verbatim}
       $ git checkout master
       # Switched to branch 'master'
       $ git merge iss53
       # Merge made by the 'recursive' strategy.
       # index.html |    1 +
       # 1 file changed, 1 insertion(+)

       $ git fetch & git merge == git pull
    \end{verbatim}
  \end{block}
\end{frame}

%-17-%
\begin{frame}[fragile]
  \frametitle{Basic Merge conflicts}
   \begin{verbatim}
   $ git merge feature-branch
   # Auto-merging index.html
   # CONFLICT (content): Merge conflict in index.html
   # Automatic merge failed; fix conflicts and then
   # commit the result.

   # <<<<<<< HEAD:index.html
   # <div id="footer">contact : email.support@github.com</div>
   # =======
   # <div id="footer">
   #  please contact us at support@github.com
   # </div>
   # >>>>>>> feature-branch:index.html
   \end{verbatim}
\end{frame}

%-18-%
\begin{frame}[fragile]
  \frametitle{oh-my-git !!!}
    \begin{block}{Abort merge}
      \begin{verbatim}
         $ git merge --abort
    \end{verbatim}
    \end{block}
    \pause
    \begin{block}{Fix files}
      \begin{verbatim}
         $ Fix conflicts manually
         $ git add files
         $ git commit
    \end{verbatim}
    \end{block}
     \pause
    \begin{block}{Checkout files}
      \begin{verbatim}
         # Suppose you have conflict with a file
         # which doesn't have any of your code in it.
         $ git checkout master -- path/to/file-name
    \end{verbatim}
    \end{block}
\end{frame}

%-19-%
\begin{frame}
  \begin{center}
    \Huge{Git rebase - interactive (Conflicts !!!)}
  \end{center}
\end{frame}

%-20-%
\begin{frame}
	\frametitle{Git Rebase}
	\begin{center}
	  \includegraphics[width=0.90\textwidth]{rebase.png}
	\end{center}
\end{frame}

%-21-%
\begin{frame}
  \frametitle{The Perils of Rebasing}
  \begin{itemize}
    \item Do not rebase commits that \alert<4>{exist outside your repository}.
    \pause
    \item Abandoning existing commits and \alert<4>{creating new ones that are similar but different}.
    \pause
    \item Collaborators will have to \alert<4>{re-merge their work} and things will get messy (If pushed to remote)
    \pause
    \item Now, to the question of whether merging or rebasing is better: hopefully you’ll see that it’s not that simple. Git is a powerful tool, and allows you to do many things to and with your history, but every team and every project is different. Now that you know how both of these things work, it’s up to you to decide which
 one is best for your particular situation.
  \end{itemize}
\end{frame}

%-22-%
\begin{frame}
  \frametitle{Rebase in action}
  \begin{itemize}
    \item Change old commit message.
    \pause
    \item Add/Remove changes to an old commit.
    \pause
    \item Squash/combine two or more commits.
    \pause
    \item Delete one or more old commits.
    \pause
    \item Split one old commit to multiple commits.
    \pause
    \item \alert{Explore more... !!!}
  \end{itemize}
\end{frame}

%-x-%
\begin{frame}
  \begin{center}
    \Huge{Git Stash - Easy work save/restore }
  \end{center}
\end{frame}

%-x-%
\begin{frame}
  \begin{center}
    \Huge{Thank You}
  \end{center}
\end{frame}

%-x-%
\begin{frame}
	\frametitle{Questions?}
	\begin{center}
	  \includegraphics[width=0.40\textwidth]{q5.jpg}
	\end{center}
\end{frame}

%----Slides----%

\end{document}
